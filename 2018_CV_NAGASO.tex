\documentclass[9pt,a4paper]{moderncv}
    \usepackage{fontspec}
    \usepackage{fontawesome}
    \usepackage{url}
    \usepackage{amsmath}
    \usepackage[T1]{fontenc}
    \usepackage[utf8]{inputenc}
   % set fonts
    %\setmainfont[
    %    BoldFont={Gill Sans Bold}, 
    %    ItalicFont={Gill Sans Italic},
    %    BoldItalicFont={Gill Sans Bold Italic}
    %]{Gill Sans Light}
    

    % moderncv themes
    \moderncvstyle{fork}
    \moderncvcolor{grey}
    
    
    % adjust the page margins
    \usepackage[scale=0.8]{geometry}
    
    % set width of the date column
    \setlength{\hintscolumnwidth}{3cm}
    
    
    \definecolor{CadetBlue}{RGB}{58,128,179}
    % color hyperlinks
    \AtBeginDocument{
        \hypersetup{colorlinks,urlcolor=CadetBlue}
    }
    
    % align name, adress and photo
    \makeatletter
    \renewcommand*{\makecvtitle}{%
      % recompute lengths (in case we are switching from letter to resume, or vice versa)
      \recomputecvlengths%
      % optional detailed information box
      \newbox{\makecvtitledetailsbox}%
      \savebox{\makecvtitledetailsbox}{%
        \addressfont\color{color2}%
        \begin{tabular}[t]{@{}r@{}}%
          \ifthenelse{\isundefined{\@addressstreet}}{}{\makenewline\addresssymbol\@addressstreet%
            \ifthenelse{\equal{\@addresscity}{}}{}{\makenewline\@addresscity}% if \addresstreet is defined, \addresscity and addresscountry will always be defined but could be empty
            \ifthenelse{\equal{\@addresscountry}{}}{}{\makenewline\@addresscountry}}%
          \ifthenelse{\isundefined{\@mobile}}{}{\makenewline\mobilesymbol\@mobile}%
          %\ifthenelse{\isundefined{\@phone}}{}{\makenewline\phonesymbol\@phone}%
          %\ifthenelse{\isundefined{\@fax}}{}{\makenewline\faxsymbol\@fax}%
          \ifthenelse{\isundefined{\@email}}{}{\makenewline\emailsymbol\emaillink{\@email}}%
          \ifthenelse{\isundefined{\@homepage}}{}{\makenewline\homepagesymbol\httplink{\@homepage}}%
          \ifthenelse{\isundefined{\@extrainfo}}{}{\makenewline\@extrainfo}%
        \end{tabular}
      }%
      % optional photo (pre-rendering)
      \newbox{\makecvtitlepicturebox}%
      \savebox{\makecvtitlepicturebox}{%
        \ifthenelse{\isundefined{\@photo}}%
        {}%
        {%
          \hspace*{\separatorcolumnwidth}%
          \color{color1}%
          \setlength{\fboxrule}{\@photoframewidth}%
          \ifdim\@photoframewidth=0pt%
            \setlength{\fboxsep}{0pt}\fi%
      \framebox{\includegraphics[width=\@photowidth]{\@photo}}}}%
      % name and title
      \newlength{\makecvtitledetailswidth}\settowidth{\makecvtitledetailswidth}{\usebox{\makecvtitledetailsbox}}%
      \newlength{\makecvtitlepicturewidth}\settowidth{\makecvtitlepicturewidth}{\usebox{\makecvtitlepicturebox}}%
      \ifthenelse{\lengthtest{\makecvtitlenamewidth=0pt}}% check for dummy value (equivalent to \ifdim\makecvtitlenamewidth=0pt)
        {\setlength{\makecvtitlenamewidth}{\textwidth-\makecvtitledetailswidth-\makecvtitlepicturewidth}}%
        {}%
      \begin{minipage}[t]{\makecvtitlenamewidth}%
        \namestyle{\@firstname\ \@familyname}%
        \ifthenelse{\equal{\@title}{}}{}{\\[0.75em]\titlestyle{\@title}}%
      \end{minipage}%
      \hfill%
      % detailed information
      \llap{%
        \begin{minipage}[t]{\makecvtitledetailswidth}%
        \vspace*{-17pt}%
        \usebox{\makecvtitledetailsbox}%
        \end{minipage}}% \llap is used to suppress the width of the box, allowing overlap if the value of makecvtitlenamewidth is forced
      % optional photo (rendering)
      \begin{minipage}[t]{\makecvtitlepicturewidth}%
        \vspace*{-17pt}%
        \vbox to 0pt{%
          \usebox{\makecvtitlepicturebox}%
        }%
      \end{minipage}\\[-0.5em]%
      % optional quote
      \ifthenelse{\isundefined{\@quote}}%
        {}%
        {{\begin{minipage}{\quotewidth}\raggedright\quotestyle{\@quote}\end{minipage}\\[2.5em]}}%
      \par}% to avoid weird spacing bug at the first section if no blank line is left after \makecvtitle
    \makeatother
    
 
    
    %
    % definitions for references
    %

    % Define \cvdoublecolumn, which sets its arguments in two columns without any labels
    % set width of the double column
    \newlength{\listdoubleitemcolumnwidth}
    \setlength{\listdoubleitemcolumnwidth}{6cm}
 
    \newcommand{\cvdoublecolumn}[2]{%
      \cvitem[0.75em]{}{%
        \begin{minipage}[t]{\listdoubleitemcolumnwidth}#1\end{minipage}%
        \hfill%
        \begin{minipage}[t]{\listdoubleitemcolumnwidth}#2\end{minipage}%
        }%
    }

    % usage: \cvreference{name}{address line 1}{address line 2}{address line 3}{address line 4}{e-mail address}{phone number}
    % Everything but the name is optional
    % If \addresssymbol, \emailsymbol or \phonesymbol are specified, they will be used.
    % (Per default, \addresssymbol isn't specified, the other two are specified.)
    % If you don't like the symbols, remove them from the following code, including the tilde ~ (space).

    \newcommand{\cvreference}[7]{%
        \textbf{#1}\newline% Name
        \ifthenelse{\equal{#2}{}}{}{\addresssymbol~#2\newline}%
        \ifthenelse{\equal{#3}{}}{}{#3\newline}%
        \ifthenelse{\equal{#4}{}}{}{#4\newline}%
        \ifthenelse{\equal{#5}{}}{}{#5\newline}%
        \ifthenelse{\equal{#6}{}}{}{\emailsymbol~\texttt{#6}\newline}%
        \ifthenelse{\equal{#7}{}}{}{\phonesymbol~#7}}

    % definition for put day month and year
    \usepackage{datetime}
	    \def\dates[#1.\monthname[#2].#3-#4.\monthname[#5].#6]{{#1}{#2}{#3}--{#4}{#5}{#6}}
    
    % personal date
    \firstname{Masaru}
    \familyname{NAGASO}
    %\title {???}
    \address{43 rue du Puits Neuf}{13100 Aix en Provence}{France}
    \mobile{+33 (0)7 83 15 24 33}
    \email{masaru.NAGASO@univ-amu.fr}
    \social[skype]{mnsaru}
    \extrainfo{ \emailsymbol\emaillink{mnsaru22@gmail.com}, \\
                \faGithub\href{https://github.com/mnagaso}{mnagaso},
                \faLinkedin\href{https://www.linkedin.com/in/nagaso-masaru-6b20705a/}{mnagaso},
               }
    
    \homepage{https://mnagaso.github.io}\vspace{10 mm}
    \photo[3cm][0pt]{myface.png}
    %\quote{Lorem ipsum dolor sit amet, consectetur adipisicing elit, sed do eiusmod tempor incididunt ut labore et dolore magna aliqua. Ut enim ad minim veniam, quis nostrud exercitation ullamco laboris nisi ut aliquip ex ea commodo consequat. Duis aute irure dolor in reprehenderit.}
    
    % uncomment to suppress automatic page numbering for CVs longer than one page
    \nopagenumbers{}

    %----------------------------------------------------------------------------------
    %            content
    %----------------------------------------------------------------------------------
    \begin{document}

   
    \maketitle
    
    \section{Experience}
    \cventry{01.oct.2018--31.mar.2019}{Scientific researcher}{}{Protisvalor/IUT-LCND, Aix-en-Procenve, France}{}{
    Numerical acoustic studies on propagation of a wave in a Sodium-cooled Fast Reactor.\newline
    Development of simulation codes/scripts package for a Spectral-element full-wave simulation and semi-automated pre/post processing. \newline
    }  % arguments 3 to 6 can be left empty
    
    \cventry{02.fev.2015--31.may.2018}{PhD candidate}{}{CEA Cadarache, Saint-Paul-lez-Durance, France}{}{
    Research and development on numerical modeling method for fluctuating acoustic medium.\newline 
    2D and 3D elasto-acoustic wave propagation simulation in Sodium-cooled fast reactors.\newline 
    }  % arguments 3 to 6 can be left empty
    
    \section{Education}
    \cventry{2015--2018}{PhD}{Aix-Marseille University}{Aix-en-Procenve, France}{}{Acoustics}  % arguments 3 to 6 can be left empty
    \cventry{2011--2014}{MSc}{The University of Tokyo}{Tokyo, Japan}{}{Ocean Technology, Policy and Environment}
    \cventry{2007--2011}{BSc}{Tokyo University of Science}{Chiba, Japan}{}{Mechanical Engineering}


    \section{PhD thesis}
    \cvitem{title}{\emph{Study of ultrasound wave propagation in a heterogeneous fluid medium for the continuous monitoring of an operating sodium-based nuclear reactor}}
    \cvitem{supervisors}{
        Dr. Dimitri Komatitsch, Prof. Joseph Moysan\newline
        (Laboratoire de mécanique et d’acoustique(LMA), CNRS/L’université d’Aix-Marseille)\newline
        Dr. Christian Lhuillier\newline
        (Le Laboratoire d'instrumentations et d'essais technologiques, CEA Cadarache)
    }
    \cvitem{description}{
        Development of three-dimensional wave propagation simulations.\newline
        Application of Spectral element method (SEM) and Finite-element time-domain (FDTD) methods.\newline
        Acousto-elastic coupling problems.
        Use of High performance computiong (HPC), CURIE@TGCC/CEA, OCCIGEN@CINES.
    }

    \section{Master thesis}
    \cvitem{title}{\emph{Development of the three-dimensional visualization and measurement method for identification of sex and species of small size fish using 25MHz-focusing acoustic probe.}}
    \cvitem{supervisors}{
        Prof. Akira Asada\newline
        (The Underwater Acoustic System Engineering Laboratory, Institute of Industrial Science, The University of Tokyo)
    }
    \cvitem{description}{
        Acoustic measurement of fish bodies using a high-frequency focusing acoustic probe.
        Development of a software for acoustic signal processing and 3D visualization of fish bodies, acoustic reflection intensity image of body surface and internal organs.\newline
        FDTD simulation of a wave propagation inside of fish bodies.\newline
        (Implementation of FDTD for elastic wave and PML damping layer.)
    }

    \section{Bachelor thesis}
    \cvitem{title}{\emph{Elasto-plastic J-integral calculation using the tetrahedral finite element model.}}
    \cvitem{supervisors}{
        Prof. Hiroshi Okada\newline
        (The Laboratory of Computational Solid Mechanics, Tokyo University of Science)
    }
    \cvitem{description}{
        Application of J-integral method to elasto-plastic FEM analysis using tetrahedral mesh.\newline
        Improvement of mesh generation software for FEM analysis\newline
        Reformation of a visualization software for the calculation models and results.
    }


    \section{Weekend projects}
    % random walk
    \cvitem{subject}{\emph{Network clustering application}}
    \cvitem{description}{
        Implementation of hierarchical network clustering code based on Map equation and Modularity.\newline
        Keywords: Map equation, Modularity, Louvain method, Page rank,
    }
     
    % Nonlinear mixed effect library
    \cvitem{subject}{\emph{The numerical computation library for estimation of subject times}}
    \cvitem{description}{
        Python library for calculating the subject time (i.e. temporal position indicating the degree of progress in a disease) written in C++ and wrapped by swig.\newline
        In the calculation routine of this library, nonlinear mixed effect modeling was implemented for calculating averaged curves of multiple bio-markers (i.e. fixed effects) and random parts which depends on each subject.\newline
        Golden search algorithm was also implemented during the routine.\newline
        This code was developed as a part of research project by Dr. Keita Tokuda, a project researcher at The University of Tokyo Hospital.\newline
        Keywords: Maximum likelihood estimation, Nonlinear mixed effect model, Golden search.
    }

    % Deep learning NLP
    \cvitem{subject}{\emph{Improvement of multi-label classification using C2AE and fine-tuning with Transformer-lm}}
    \cvitem{description}{
        In order to improve the accuracy of multi-label classification task with Canonical Correlated AutoEncoder (C2AE) for limited amount of input texts, we applied the method of "Improving Language Understanding by Generative Pre-Training" so called (finetune-transformer-lm). \newline
        Keywords: Natural language processing, Deep Learning, multi-label classification, Transformer, Language model, C2AE
    }   
 
    % Word-net generation and analysis scripts
    \cvitem{subject}{\emph{Generation of semantic networks with review texts of popular products and generation of learning model for creating a new hit product}}
    \cvitem{description}{
        This is a part of another research project on "computational creativity" by Dr. Akihito Sudo, a researcher/research manager at The University of Shizuoka.\newline
        First, we generates two semantic networks, one is generated from reviews texts written for a hit product and another is from reviews for multiple products in the category which the target product belongs to.\newline
        By using these semantic networks and difference between them as a data set, we are trying to generate learning model to generate keywords for the next hit products.\newline
        Keywords: Natural language processing, word2vec, Semantic network, Machine learning, SMOTE.
    }
    % web scraper
    \cvitem{subject}{\emph{Web scraping scripts}}
    \cvitem{description}{
        This is a set of scraping scripts developed for gathering review texts from Amazon.com for generating semantic networks concerning the above project.\newline
        A python library "Scrapy" was used as the engine of scraping spiders.
        Keywords: Web Scraping, 
    }
        


    % force page break
    \pagebreak
    
    \section{Technical skills}
    \cvline{Operating systems}{Linux, OS X, Windows, Slurm}
    \cvline{Programming languages etc.}{C, C++, C\#, Fortran, Python, Swig, MPI, OpenMP, VTK, HDF5, Chuck, Markdown, \LaTeX{}}
    \cvline{Web tools}{xhtml, css, JavaScript, Node.js, MySQL, Mongodb}
    \cvline{Development environments}{Docker, Vim, Visual Studio Code, Git, SVN, Redmine, Bitbucket}
    \cvline{Analysis tools}{Jupyter notebook (lab), Matplotlib, Holoviews}
    \cvline{Other softwares}{Microsoft Word, Excel, PowerPoint, Adobe Photoshop, Adobe Illustrator, Gimp, Inkscape}
    \cvline{Music theory}{Knowledge and experiences of modal/codal music}
    \cvline{Instruments}{Piano, Hammond Organ, Synthesizer, Saxophones}


    \section{Languages}
    \cvlistitem{Japanese (First language)}
    \cvlistitem{English (Fluent)}
    \cvlistitem{French (Basic)}

    \section{Conferences}
    \cvline{2017}{8th ANNIMA (International conference on Advancements in Nuclear Instrumentation Measurement Methods and their Applications) in Liège. Poster session.}
    \cvline{2016}{19th WCNDT (World Conference on Non-Destructive Testing) in Munich. Oral session.}
    \cvline{2013}{Oceans ’13 MTS/IEEE in San Diego. Student Poster Competition.}

    \nocite{*}
    \bibliographystyle{plain}
    \bibliography{my_pubs.bib}      
    
    \section{References}

    \subsection{}
    \cvdoublecolumn{\cvreference{Dr. Dimitri Komatitsch}
    {Laboratory of Mechanics and Acoustics}
    {CNRS Marseille}
    {CNRS LMA UMR 7031, Bureau 120, 4 impasse Nikola Tesla, CS 40006}
    {13453 Marseille cedex 13, France}
    {komatitsch@lma.cnrs-mrs.fr}
    {+33 4 84 52 42 52}%
    }
    {\cvreference{Prof. Joseph Moysan}
    {Laboratory of Mechanics and Acoustics}
    {Aix-Marseille University}
    {LMA UMR 7031 site LCND, 413 Avenue Gaston Berger}
    {13625 Aix-en-Provence, France}
    {joseph.moysan@univ-amu.fr}
    {+33 4 42 93 90 52}%
    }

    \subsection{}
    \cvdoublecolumn{\cvreference{Dr. Katsunori Mizuno}
    {Department of Environment systems}
    {The University of Tokyo}
    {5-1-5, Kashiwanoha, Kashiwa city}
    {277-8561, Japan}
    {kmizuno@edu.k.u-tokyo.ac.jp}
    {+81 4 7136 4697}%
    }
    {\cvreference{Prof. Hiroshi Okada}
    {Department of Mechanical Engineering, Faculty of Science and Technology}
    {Tokyo University of Science}
    {2641 Yamazaki, Noda-shi, Chiba-ken}
    {278-8510, Japan}
    {hokada@rs.noda.tus.ac.jp}
    {+81 4 7124 1501, ext:3922}%
    } 
  
  
    %\clearpage
    %%-----       letter       ---------------------------------------------------------
    %% recipient data


    %% recipient data
    %\recipient{Center for Ultrasound Research \& Translation\\Massachusetts General Hospital}{101 Merrimac Street, 3rd Floor\\Boston, MA 02114}
    %\date{July 03, 2018}
    %\opening{Dear Dr. Anthony Samir}
    %\closing{Yours faithfully,}
    
    %\makelettertitle

  
    %I appreciate your considering my application for the position “Post-Doctoral Scholar/ Research Scientist - MGH Center for Ultrasound Research and Translation”. My background and skills in laboratory techniques will prove to be an effective match for your qualifications requirements.
    
    %My main academic/engineering interests are on acoustics and numerical analysis. The interest in acoustics have been cultivated through my experiences of Jazz studies which is one of my important lifeworks.
    %In order to study another aspect of “sound”,  as my master’s research, I learned underwater (ocean) acoustics and developed an acoustic CT scanning method for sex/species detection using 25 MHz focusing probe. Development of numerical codes for signal processing and 3D image reconstruction are also important experiences which may be great helps for this postdoctoral position which concerns acoustic imaging.
    
    %In order to expand my experience on acoustics and numerical analysis, I carried out (mainly) numerical studies on wave propagation in a cooling circuit of Sodium-cooled Fast Reactors with French Atomic Commission (CEA) and French Centre National de la Recherche Scientifique (CNRS).  During this Ph.D.  I experienced development of a numerical code for wave propagation simulation using Spectral Element Method, which is a higher-order finite element method.
    %For the calculations, we used two french super computers (CURIE at Très Grand Centre de calcul du CEA, OCCIGEN at Centre Informatique National de l’Enseignement Supérieur). As the numerical tool for calculation of wave propagation, I carried out further development of SPECFEM2D/3D (\url{https://geodynamics.org/cig/software/specfem3d/}) which have been developed and used for High Performance Computing (HPC) environment (which uses GPGPU, MPI, vectorization and Fortran).
    
    %Through these studies, I obtained knowledge and experiences of operations and development of scientific numerical codes under linux environment including HPC.
    
    %Through the private projects, I experienced not only numerical computation techniques of acoustics but also implementations of algorithms belonging to other scientific domains e.g. network clustering, nonlinear mixed effect model and machine learning. 
    %Please find some details of those works in my CV at the “private development” section.
    
    %My career goal is to be a researcher who studies multidisciplinary subjects between acoustics and computational techniques.
    %The research subject of this position is one of modern applications of acoustic techniques, thus this is exactly what I want to pursuit in order to expand my academic/engineering interests.
    
    %Thank you for your consideration. I would be grateful for the opportunity to speak with you in person regarding my qualifications for this position; please let me know if I can provide you with any additional information.
  
    %\makeletterclosing

    \end{document}
